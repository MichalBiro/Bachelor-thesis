\chapter*{Úvod}
\addcontentsline{toc}{chapter}{Úvod}

Oblasť automatizácie a automatického riadenia sa rýchlo rozvíja a v modernom svete má veľké zastúpenie v oblasti výroby. Vo svete sú čoraz viac využívané automatické výrobné linky, prevádzky a závody, práve kvôli zvyšovaniu produktivity a redukciou vznikajúcich chýb vo výrobe. Aby mohol rozvoj napredovať je potrebné pre danú oblasť podporovať vzdelávanie budúcich technikov a vedcov. Práve kvalita vzdelávania sa totiž výrazne odrazí v nasledujúcich rokoch ich práce. 

Hoci dôležitou časťou vzdelávania je práve teória riadenia, s ktorou sa študenti univerzít stretávajú na prednáškach, platí, že človek si najviac zapamätá, keď si to môže aj vyskúšať v praxi.  Nie každý má však túto možnosť a nie každá univerzita disponuje rovnako vybavenými laboratóriami. Preto je pre mnohých niekedy takmer nemožné stretnúť sa počas štúdia s praxou a to hlavne v časoch, kedy sa študenti nemôžu  zúčastniť prezenčnej vzdelávacej formy. Absencia praktických ukážok môže mať za následok neskoršie problémy pri prechode študentov na pracovný trh. 

AutomationShield so svojimi zariadeniami prináša jednoduché riešenie tohto problému, kedy tvorbou zariadení minimálnych rozmerov a nákladov, poskytuje študentom možnosť vyskúšať si riadenie systému v praxi. Vďaka svojej cene nie sú pre školu tieto zariadenia finančne náročné a svojimi šikovnými rozmermi poskytujú študentom možnosť pracovať na nich aj z domova., čo je viac ako vítané pri dištančnej forme štúdia. Momentálne sa v projekte nachádza 11 zariadení a BOBShield je jedným z nich. Zariadenie už prešlo určitým vývojom a momentálne existujú 2 verzie. Hoci pri druhom zariadení prišlo ku viacerým zmenám a vylepšeniam oproti prvej verzii, stále poskytuje priestor na vylepšenie. Cieľom práce bolo opäť urobiť analýzu zariadenia a prísť s nápadmi na vylepšenie, či už hardvéru alebo softvéru. V oblasti softvéru išlo o vývoj API v programovacom prostredí Simulink a tvorba príkladu na riadenie systému. Pri hardvéry sme sa zamerali na použité komponenty, kde sme hľadali ich varianty, nachádzajúce sa na trhu a analyzovali sme ako by ich zmena vplývala na chod zariadenia.

Túto tému som si vybral hlavne z dôvodu   . Tiež považujem celú myšlienku AutomationShieldu veľmi efektívnu a praktickú. Páči sa mi snaha poskytnúť študentom priestor pre rozvoj a vzdelávanie nie len v teoretickej rovine ale aj v praxi. Hoci ide o v princípe jednoduché zariadenia, každé má veľký potenciál a proces vývoja a vylepšovania jednotlivých verzií a prevedení so sebou prináša taktiež drahocenné skúsenosti a vedomosti.      

