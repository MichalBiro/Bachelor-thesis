\chapter{BoBshield API}
\label{kap:3}

\section{API pre MATLAB}
\label{kap:3.1}
MATLAB je programovacie a výpočtové prostredie, využívané širokou škálou technikov a vedcov na numerické výpočty, modelovanie, analýzu dát a ich interpretáciu, návrhy algoritmov, simuláciu a riadenie systémov. Názov vznikol skrátením anglického výrazu Matrix Laboratory, z čoho vyplýva, že základom pri výpočtoch je práve práca s maticami. Jeho využitie má veľmi široký záber a jeho uplatnenie vieme nájsť aj pri robotike alebo umelej inteligencii. Vďaka širokému spektru toolboxov a balíkov voľne dostupných na platformách ako je Github sa jeho obzory stále zväčšujú. MATLAB je pre univerzity a študentov voľne dostupný, preto je medzi študentami s technickým zameraním veľmi populárny a často využívaný. Aby si študenti mohli vyskúšať automatické riadenie systému gulička na tyči (Ball on Beam) je potrebné vytvoriť k nášmu hardware aj príslušné API pre daný software.  

\subsection{Tvorba API pre MATLAB}
\label{kap:3.1.1}

\subsection{Knižnica pre snímač}
\label{kap:3.1.2}

\subsection{Funkcie}
\label{kap:3.1.3}

\subsection{Príklady}
\label{kap:3.1.4}

\section{API pre Simulink}
\label{kap:3.2}

